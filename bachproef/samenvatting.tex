\IfLanguageName{english}{%
\selectlanguage{dutch}
\chapter*{Samenvatting}

\selectlanguage{english}
}{}

\chapter*{\IfLanguageName{dutch}{Samenvatting}{Abstract}}

Toegepaste Informatica in HoGent kan helaas niet beschikken over alle mogelijke technologieën die in de IT-wereld plaats vinden. Toch zijn er bepaalde onderwerpen die de basiskennis van de studenten  zee goed kan versterken. Na het aanleren van Kubernetes en Docker is de monitoring van deze tools de logische volgende stap. Monitoring binnen de Kubernetes en Docker omgeving is om vele reden belangrijk en aan de hand van dit onderzoek zal dat proberen bewezen worden.

De eerste stap in de bachelorproef is een situatie schetsen waarin de IT en HoGent zich momenteel bevinden. Het onderzoek focust zich op het ontstaan van de virtualisatie en gecontaineriseerde omgeving en de bestaansreden van monitoring in zo een omgeving. 

Via de methodologie zal er onderzocht worden welke monitoringtools daadwerkelijk geschikt zijn om te gebruiken binnen het lessenpakket. De MoSCoW-analyse zal hiervoor de beste onderzoeksmethode zijn. 

Na het vinden van een geschikte tool, zal hiervoor een gepaste Proof-of-Concept opgesteld worden, dat kan worden gerecreëerd door de studenten Toepaste Informatica, waarmee zijn dan aan de slag kunnen om de wereld van monitoring te ontdekken. 

Uit het onderzoek voor het vinden van een geschikte monitoringtool zijn twee tools overgebleven: Sysdig, Zabbix en Prometheus. Uiteindelijk werd beslist om te kiezen voor Prometheus op vertrouwen van de community achter de tools. 