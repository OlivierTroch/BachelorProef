\chapter{\IfLanguageName{dutch}{Inleiding}{Introduction}}
\label{ch:inleiding}


\section{\IfLanguageName{dutch}{Probleemstelling}{Problem Statement}}
\label{sec:probleemstelling}

Studenten binnen de opleiding toegepaste informatica in HoGent zijn niet bekend met technologieën en tools die gebruikt worden binnen het Kubernetes en Docker domein. Het monitoren van deze technologieën is daar een onderdeel van. Ten eerste zijn Kubernetes en Docker geen eenvoudige technologieën om te implementeren en ten tweede is er een groot aanbod van mogelijke tools die hiervoor gebruikt kan worden en weten we niet welke voor deze doelgroep geschikt is. Om dit onderzoek de ondersteunen zal er beroep gedaan worden op de bachelorproef van Detemmerman Cedric waarvan de opstart van Docker en Minikube (lightweight versie van Kubernetes) zal worden gebruikt. 


\section{\IfLanguageName{dutch}{Onderzoeksvraag}{Research question}}
\label{sec:onderzoeksvraag}

Zoals eerder vermeld is er een groot aanbod aan monitoringtools voor een Docker/Kubernetes environment. Om die reden zal dit onderzoek nagaan welke tools voor de studenten geschikt zijn. Zo bekomen we volgende hoofdonderzoeksvraag:

\begin{itemize}
    \item Welke monitoringtool voor een Kubernetes/Docker environment is het meest geschikt voor de studenten toegepaste informatica om een goede basiskennis te bekomen.
\end{itemize}

Eveneens zijn er nog een aantal deelvragen die ook belangrijk zijn in dit onderzoek:

\begin{itemize}
    \item Wat zijn de belangrijkste verschillen tussen de tools?
    \item Waar moeten de tools inzicht op geven?
    \item Welke Proof-of-Concept is voldoende om de tool aan te leren?
\end{itemize}

\section{\IfLanguageName{dutch}{Onderzoeksdoelstelling}{Research objective}}
\label{sec:onderzoeksdoelstelling}

Dit onderzoek heeft als doel het vinden van een geschikte monitoringtool die voor de studenten toegepaste informatica een meerwaarde kan betekenen. Door dit onderzoek zal zowel een vergelijking van tools beschikbaar zijn, alsook een reproduceerbare proof-of-concept die studenten aan de hand van documentatie kunnen bekomen. Aan de hand van dit proof-of-concept kunnen de educatieve doeleinden bereikt worden.

\section{\IfLanguageName{dutch}{Opzet van deze bachelorproef}{Structure of this bachelor thesis}}
\label{sec:opzet-bachelorproef}

De rest van deze bachelorproef is als volgt opgebouwd:

In Hoofdstuk~\ref{ch:stand-van-zaken} wordt een overzicht gegeven van de stand van zaken binnen het onderzoeksdomein, op basis van een literatuurstudie.

In Hoofdstuk~\ref{ch:methodologie} wordt de methodologie toegelicht en worden de gebruikte onderzoekstechnieken besproken om een antwoord te kunnen formuleren op de onderzoeksvragen.

In Hoofdstuk~\ref{ch:conclusie}, tenslotte, wordt de conclusie gegeven en een antwoord geformuleerd op de onderzoeksvragen. Daarbij wordt ook een aanzet gegeven voor toekomstig onderzoek binnen dit domein.