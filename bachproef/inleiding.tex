\chapter{\IfLanguageName{dutch}{Inleiding}{Introduction}}
\label{ch:inleiding}


\section{\IfLanguageName{dutch}{Probleemstelling}{Problem Statement}}
\label{sec:probleemstelling}

Studenten binnen de opleiding Toegepaste Informatica in HoGent zijn niet bekend met technologieën en tools die gebruikt worden binnen het Kubernetes en Docker domein. Het monitoren van deze technologieën is daar een onderdeel van. Ten eerste zijn Kubernetes en Docker geen eenvoudige technologieën om te implementeren en daarvoor zal er beroep gedaan worden op de bachelorproef van Detemmerman Cedric. Ten tweede is er een groot aanbod van mogelijke tools die hiervoor gebruikt kan worden en weten we niet welke voor deze doelgroep geschikt is.


\section{\IfLanguageName{dutch}{Onderzoeksvraag}{Research question}}
\label{sec:onderzoeksvraag}

Zoals eerder vermeld is er een groot aanbod aan monitoringtools voor een Kubernetes/Docker environment. Om die reden zal dit onderzoek nagaan welke tools voor de studenten geschikt zijn. Zo bekomen we volgende hoofdonderzoeksvraag:

\begin{itemize}
    \item Welke monitoringtool voor een Kubernetes/Docker environment is het meest geschikt voor de studenten Toegepaste Informatica om zo een goede basiskennis te bekomen.
\end{itemize}

Eveneens zijn er nog een aantal deelvragen die ook belangrijk zijn in dit onderzoek:

\begin{itemize}
    \item Wat zijn de belangrijkste verschillen tussen de tools?
    \item Waar moeten de tools inzicht op geven?
    \item Welke Proof-of-Concept is voldoende om de tool aan te leren?
\end{itemize}

%Wees zo concreet mogelijk bij het formuleren van je onderzoeksvraag. Een onderzoeksvraag is trouwens iets waar nog niemand op dit moment een antwoord heeft (voor zover je kan nagaan). Het opzoeken van bestaande informatie (bv. ``welke tools bestaan er voor deze toepassing?'') is dus geen onderzoeksvraag. Je kan de onderzoeksvraag verder specifiëren in deelvragen. Bv.~als je onderzoek gaat over performantiemetingen, dan 

\section{\IfLanguageName{dutch}{Onderzoeksdoelstelling}{Research objective}}
\label{sec:onderzoeksdoelstelling}

Dit onderzoek heeft tot doel om een geschikte monitoringtool te vinden die voor de studenten Toegepaste Informatica een meerwaarde kan betekenen. Door dit onderzoek zal zowel een vergelijking van tools beschikbaar zijn, alsook een reproduceerbare proof-of-concept die studenten aan de hand van een handleiding kunnen bekomen. Aan de hand van dit proof-of-concept kunnen de educatieve doeleinden bereikt worden.
%Wat is het beoogde resultaat van je bachelorproef? Wat zijn de criteria voor succes? Beschrijf die zo concreet mogelijk. Gaat het bv. om een proof-of-concept, een prototype, een verslag met aanbevelingen, een vergelijkende studie, enz.

\section{\IfLanguageName{dutch}{Opzet van deze bachelorproef}{Structure of this bachelor thesis}}
\label{sec:opzet-bachelorproef}

% Het is gebruikelijk aan het einde van de inleiding een overzicht te
% geven van de opbouw van de rest van de tekst. Deze sectie bevat al een aanzet
% die je kan aanvullen/aanpassen in functie van je eigen tekst.

De rest van deze bachelorproef is als volgt opgebouwd:

In Hoofdstuk~\ref{ch:stand-van-zaken} wordt een overzicht gegeven van de stand van zaken binnen het onderzoeksdomein, op basis van een literatuurstudie.

In Hoofdstuk~\ref{ch:methodologie} wordt de methodologie toegelicht en worden de gebruikte onderzoekstechnieken besproken om een antwoord te kunnen formuleren op de onderzoeksvragen.

% TODO: Vul hier aan voor je eigen hoofstukken, één of twee zinnen per hoofdstuk

In Hoofdstuk~\ref{ch:conclusie}, tenslotte, wordt de conclusie gegeven en een antwoord geformuleerd op de onderzoeksvragen. Daarbij wordt ook een aanzet gegeven voor toekomstig onderzoek binnen dit domein.