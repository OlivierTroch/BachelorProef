\chapter{Conclusie}
\label{ch:conclusie}

Op het begin van het onderzoek werden een aantal onderzoeksvragen opgesteld waaronder één hoofdonderzoeksvraag en meerdere deelvragen:

\begin{itemize}
    \item Welke monitoringtools zijn geschikt?
    \item Wat zijn de belangrijkste verschillen tussen de gekozen tools?
    \item Waar moet de gekozen tool inzicht op geven?
    \item Welk proof-of-concept is genoeg om de leerstof te verstaan?
\end{itemize}

De hoofdonderzoeksvraag van dit onderzoek was het vinden van een tool die geschikt was om de studenten toegepaste informatie genoeg bij te leren over de mogelijkheden in een monitoringtool voor een Docker/Kubernetes omgeving. Dit betekende dat de tool zowel technische mogelijkheden nodig had, alsook educatieve.

Door een requirements-analyse uit te voeren en hierin de belangrijkste verschillen bloot te leggen werden hieruit drie tools geschikt verklaard:

\begin{itemize}
    \item Sysdig
    \item Zabbix
    \item Prometheus
\end{itemize}

De drie tools waren praktisch hetzelfde op basis van de requirements-analyse waardoor de keuze werd gemaakt door de monitoringcommunity, hierbij was Prometheus de tool die met voorsprong uitblonk. Voor de Proof-of-Concept werd dan uiteindelijk ook voor die tool gekozen.

Er werd een aantal dashboards overlopen die standaard geïnstalleerd werden waarin verschillende metrics zichtbaar werden gesteld. Het importeren van json files voor community created dashboards werd ook uitgelegd omdat daar uit snel een dashboard kan worden opgezet. Daarna werd ook een eigen dashboard gecreëerd om aan te tonen dat het maken van dashboards vrij eenvoudig en interactief was. Op deze dashboards werden dan een aantal technieken toegepast om aan te tonen dat bij gebruik van een pod, deze zeker aanpassen.  

De community van talloze experts die het opensource project ondersteunen zorgden ervoor dat de installatie van de tool vrij eenvoudig op te zetten was. Zonder deze tools was de moeilijkheidsgraad voor het opzetten van Prometheus eventueel al een reden genoeg om een andere tool te kiezen. Doordat Prometheus een befaamde tool was die in vele bedrijven gebruikt werd, was er al een vermoeden dat deze tool in de laatste vergelijking zou terecht komen. Door het gebruik van een containeromgeving kon er gemakkelijk een andere pod toegevoegd worden om die dan alsook te monitoren in Prometheus. Dit wil zeggen dat de tool genoeg inzicht gaf op alles wat nodig was om de onderzoeksvragen te beantwoorden.

Hopelijk werd dit onderzoek gezien als een meerwaarde voor het lessenpakket bij de studenten toegepaste informatica zodat monitoring voor gecontaineriseerde omgevingen basiskennis werd. Aangezien de populariteit van deze technologie alleen maar steeg kon dit zeker een meerwaarde zijn, al was het enkel voor de lezer van dit onderzoek.