%==============================================================================
% Sjabloon onderzoeksvoorstel bachelorproef
%==============================================================================
% Gebaseerd op LaTeX-sjabloon ‘Stylish Article’ (zie voorstel.cls)
% Auteur: Jens Buysse, Bert Van Vreckem
%
% Compileren in TeXstudio:
%
% - Zorg dat Biber de bibliografie compileert (en niet Biblatex)
%   Options > Configure > Build > Default Bibliography Tool: "txs:///biber"
% - F5 om te compileren en het resultaat te bekijken.
% - Als de bibliografie niet zichtbaar is, probeer dan F5 - F8 - F5
%   Met F8 compileer je de bibliografie apart.
%
% Als je JabRef gebruikt voor het bijhouden van de bibliografie, zorg dan
% dat je in ``biblatex''-modus opslaat: File > Switch to BibLaTeX mode.

\documentclass{voorstel}

\usepackage{lipsum}

%------------------------------------------------------------------------------
% Metadata over het voorstel
%------------------------------------------------------------------------------

%---------- Titel & auteur ----------------------------------------------------

% TODO: geef werktitel van je eigen voorstel op
\PaperTitle{Titel voorstel}
\PaperType{Onderzoeksvoorstel Bachelorproef 2018-2019} % Type document

% TODO: vul je eigen naam in als auteur, geef ook je emailadres mee!
\Authors{Steven Stevens\textsuperscript{1}} % Authors
\CoPromotor{Piet Pieters\textsuperscript{2} (Bedrijfsnaam)}
\affiliation{\textbf{Contact:}
  \textsuperscript{1} \href{mailto:steven.stevens.u1234@student.hogent.be}{steven.stevens.u1234@student.hogent.be};
  \textsuperscript{2} \href{mailto:piet.pieters@acme.be}{piet.pieters@acme.be};
}

%---------- Abstract ----------------------------------------------------------

\Abstract{Hier schrijf je de samenvatting van je voorstel, als een doorlopende tekst van één paragraaf. Wat hier zeker in moet vermeld worden: \textbf{Context} (Waarom is dit werk belangrijk?); \textbf{Nood} (Waarom moet dit onderzocht worden?); \textbf{Taak} (Wat ga je (ongeveer) doen?); \textbf{Object} (Wat staat in dit document geschreven?); \textbf{Resultaat} (Wat verwacht je van je onderzoek?); \textbf{Conclusie} (Wat verwacht je van van de conclusies?); \textbf{Perspectief} (Wat zegt de toekomst voor dit werk?).

Bij de sleutelwoorden geef je het onderzoeksdomein, samen met andere sleutelwoorden die je werk beschrijven.

Vergeet ook niet je co-promotor op te geven.
}

%---------- Onderzoeksdomein en sleutelwoorden --------------------------------
% TODO: Sleutelwoorden:
%
% Het eerste sleutelwoord beschrijft het onderzoeksdomein. Je kan kiezen uit
% deze lijst:
%
% - Mobiele applicatieontwikkeling
% - Webapplicatieontwikkeling
% - Applicatieontwikkeling (andere)
% - Systeembeheer
% - Netwerkbeheer
% - Mainframe
% - E-business
% - Databanken en big data
% - Machineleertechnieken en kunstmatige intelligentie
% - Andere (specifieer)
%
% De andere sleutelwoorden zijn vrij te kiezen

\Keywords{Onderzoeksdomein. Keyword1 --- Keyword2 --- Keyword3} % Keywords
\newcommand{\keywordname}{Sleutelwoorden} % Defines the keywords heading name

%---------- Titel, inhoud -----------------------------------------------------

\begin{document}

\flushbottom % Makes all text pages the same height
\maketitle % Print the title and abstract box
\tableofcontents % Print the contents section
\thispagestyle{empty} % Removes page numbering from the first page

%------------------------------------------------------------------------------
% Hoofdtekst
%------------------------------------------------------------------------------

% De hoofdtekst van het voorstel zit in een apart bestand, zodat het makkelijk
% kan opgenomen worden in de bijlagen van de bachelorproef zelf.
%---------- Inleiding ---------------------------------------------------------

\section{Introductie} % The \section*{} command stops section numbering
\label{sec:introductie}
Containertechnologie vindt in recordtempo zijn weg naar de data-omgeving van de ondernemingen.\\~\autocite{cole2016} Het gemak waarmee containerplatformen zoals Docker kunnen worden ingezet, suggereert dat ze de dominantste archictectuur zijn en zullen blijven voor deze en de volgende generatie services en microservices.
\\
De uitdaging om een goede monitoring van containers op te stellen is belangrijk. Zoals te verwachten is, zullen traditionele monitoringsplatforms die vooral gebasseerd zijn op virtualisatie, niet voldoende zijn. Containers zijn zeer vluchtig in onstaan en net zo snel in het verdwijnen. Die snelheid is vaak te danken aan een geautomatisseerd proces. Deze bevinden zich tussen host -en applicatielaag, wat het moeilijk maakt om in detail te zien hoe ze zich gedragen en of ze efficiënt gebruik maken van hulpbronnen (e.g. CPU, RAM, etc.) en goede systeem prestaties leveren. De valkuil voor vele bedrijven is dat deze denken dat omdat een container slechts een mini-host is, een eenvoudige hostmonitoring voldoende is. Dit idee wordt echter al snel uit de weg geruimd, omdat het aantal containers snel begint te vegroten wat een traditionele hostmonitoring niet kan bijhouden.
\\
\\
Binnen HoGent is er voorlopig geen opleidingsonderdeel omtrent monitoring, specifiek op een docker omgeving in een Kubernetes orkestratie. Daar dit een zeer interessant en relevant onderwerp is, is het een goede zaak om deze leerstof bij te leren aan de toekomstige studenten van HoGent. De onderzoeksvraag komt in dit geval uit van een docent tewerkgesteld in HoGent. Omdat er nog geen intern onderzoek heeft plaatsgevonden, is deze onderzoeksvraag dus onstaan. 
\\
De doelstelling van dit onderzoek is bepalen welke monitoringtool hiervoor geschikt is en welke haalbaar is voor de studenten. 
\\
Zo bekomen we volgende onderzoeksvragen:
\begin{itemize}
    \item Welke monitoringtools zijn geschikt?
    \item Wat zijn de belangrijkste verschillen tussen de gekozen tools?
    \item Waar moet de gekozen tool inzicht op geven?
    \item Welke alerts zijn belangrijk en welke niet?
    \item Welk proof-of-concept is genoeg om de leerstof te verstaan?
\end{itemize}

%---------- Stand van zaken ---------------------------------------------------

\section{State-of-the-art}
\label{sec:state-of-the-art}

Hier beschrijf je de \emph{state-of-the-art} rondom je gekozen onderzoeksdomein. Dit kan bijvoorbeeld een literatuurstudie zijn. Je mag de titel van deze sectie ook aanpassen (literatuurstudie, stand van zaken, enz.). Zijn er al gelijkaardige onderzoeken gevoerd? Wat concluderen ze? Wat is het verschil met jouw onderzoek? Wat is de relevantie met jouw onderzoek?

Verwijs bij elke introductie van een term of bewering over het domein naar de vakliteratuur, bijvoorbeeld~\autocite{Doll1954}! Denk zeker goed na welke werken je refereert en waarom.

% Voor literatuurverwijzingen zijn er twee belangrijke commando's:
% \autocite{KEY} => (Auteur, jaartal) Gebruik dit als de naam van de auteur
%   geen onderdeel is van de zin.
% \textcite{KEY} => Auteur (jaartal)  Gebruik dit als de auteursnaam wel een
%   functie heeft in de zin (bv. ``Uit onderzoek door Doll & Hill (1954) bleek
%   ...'')

Je mag gerust gebruik maken van subsecties in dit onderdeel.

%---------- Methodologie ------------------------------------------------------
\section{Methodologie}
\label{sec:methodologie}

Hier beschrijf je hoe je van plan bent het onderzoek te voeren. Welke onderzoekstechniek ga je toepassen om elk van je onderzoeksvragen te beantwoorden? Gebruik je hiervoor experimenten, vragenlijsten, simulaties? Je beschrijft ook al welke tools je denkt hiervoor te gebruiken of te ontwikkelen.

%---------- Verwachte resultaten ----------------------------------------------
\section{Verwachte resultaten}
\label{sec:verwachte_resultaten}

Hier beschrijf je welke resultaten je verwacht. Als je metingen en simulaties uitvoert, kan je hier al mock-ups maken van de grafieken samen met de verwachte conclusies. Benoem zeker al je assen en de stukken van de grafiek die je gaat gebruiken. Dit zorgt ervoor dat je concreet weet hoe je je data gaat moeten structureren.

%---------- Verwachte conclusies ----------------------------------------------
\section{Verwachte conclusies}
\label{sec:verwachte_conclusies}

Hier beschrijf je wat je verwacht uit je onderzoek, met de motivatie waarom. Het is \textbf{niet} erg indien uit je onderzoek andere resultaten en conclusies vloeien dan dat je hier beschrijft: het is dan juist interessant om te onderzoeken waarom jouw hypothesen niet overeenkomen met de resultaten.



%------------------------------------------------------------------------------
% Referentielijst
%------------------------------------------------------------------------------
% TODO: de gerefereerde werken moeten in BibTeX-bestand ``voorstel.bib''
% voorkomen. Gebruik JabRef om je bibliografie bij te houden en vergeet niet
% om compatibiliteit met Biber/BibLaTeX aan te zetten (File > Switch to
% BibLaTeX mode)

\phantomsection
\printbibliography[heading=bibintoc]

\end{document}
